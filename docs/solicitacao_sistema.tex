\documentclass[12pt,a4paper]{article}
\usepackage[utf8]{inputenc}
\usepackage[brazil]{babel}
\usepackage{geometry}
\geometry{margin=2.5cm}

\title{\textbf{Solicitação do Sistema}\\Repositório Digital Institucional UEMA}
\author{Universidade Estadual do Maranhão}
\date{Dezembro 2025}

\begin{document}
\maketitle

\section{Identificação}
\begin{itemize}
    \item \textbf{Solicitante:} Pró-Reitorias UEMA
    \item \textbf{Data:} Dezembro 2025
    \item \textbf{Prioridade:} Alta
\end{itemize}

\section{Descrição da Necessidade}
Os setores da universidade enfrentam dificuldades na gestão de documentos e processos:
\begin{itemize}
    \item Documentos armazenados em locais dispersos
    \item Busca manual e demorada
    \item Falta de controle de versões
    \item Processos sem rastreabilidade
\end{itemize}

\section{Requisitos de Negócio}
\begin{enumerate}
    \item Centralizar documentos de todos os setores
    \item Permitir busca rápida por título, conteúdo e tags
    \item Controlar acesso por perfil de usuário
    \item Acompanhar status de processos em tempo real
    \item Gerar relatórios gerenciais
\end{enumerate}

\section{Valor Agregado}
\begin{itemize}
    \item \textbf{Eficiência:} Redução do tempo de busca
    \item \textbf{Qualidade:} Padronização de documentos
    \item \textbf{Segurança:} Controle de acesso e backup
    \item \textbf{Inovação:} Assistente de IA para consultas
\end{itemize}

\section{Restrições}
\begin{itemize}
    \item Orçamento limitado (uso de tecnologias gratuitas)
    \item Prazo: semestre letivo
    \item Hospedagem em nuvem (Vercel + Supabase)
\end{itemize}

\section{Critérios de Aceitação}
\begin{enumerate}
    \item Sistema acessível via navegador web
    \item Login funcional com diferentes níveis de acesso
    \item Upload e download de documentos
    \item Busca por palavras-chave
    \item Relatórios exportáveis
\end{enumerate}

\end{document}
